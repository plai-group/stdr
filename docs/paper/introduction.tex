
In order to compensate for epistemic uncertainty due to modelling approximations and unmodelled aleatoric uncertainty, deterministic simulators are often ``converted'' to ``stochastic'' simulators by perturbing the state at each time step.
In practice this allows the simulator to explain the variability observed in real data without requiring excessive observation noise.
Such models are more resilient to misspecification, are capable of providing uncertainty estimates, and provide better inferences in general~\citep{moller2011parameter, Saarinen2008hh, LV200874, stochchem, renard2013stochastic}.

Often, state-independent Gaussian noise with heuristically tuned variance is used to perturb the state~\citep{adhikari2013introductory, brockwell2016introduction, fox1997stochastic, reddy2016simulating, du2006dynamics, allen2017primer, Mbalawata2013}.
However, naively adding noise to the state will, in many applications, render the perturbed input state ``invalid,''
where invalid states cause the simulator to raise an exception and not return a value~\citep{RAZAVI201995, lucas2013failure, gmd2019crash}. 
We formally define failure by extending the possible output of the simulator to include $\bot$ (read as ``bottom'') denoting simulator failure.
The principal contribution of this paper is a technique for avoiding invalid states by choosing perturbations that minimize the failure rate.
The technique we develop results in a reduction in simulator failures, while maintaining the original model.

Examples of failure modes include ordinary differential equation (ODE) solvers not converging to the required tolerance in the allocated time, or, the perturbed state entering into an unhandled configuration, such as solid bodies intersecting.
Establishing the state-perturbation pairs that cause failure is non-trivial.
Hence, the simulation artifact can be sensitive to seemingly inconsequential alterations to the state -- a property we describe as ``brittle.'' 
Failures waste computational resources and reduce the diversity of simulations for a finite sample budget, for instance, when used as the proposal in sequential Monte Carlo.
As such, we wish to learn a proposal over perturbations such that the simulator exits with high probability, but renders the joint distribution unchanged.

We proceed by framing sampling from brittle simulators as rejection samplers, then seek to eliminate rejections by estimating the state-dependent density over perturbations that do not induce failure.
We then demonstrate that using this learned proposal yields lower variance results when used in posterior inference with a fixed sample budget, such as pseudo-marginal evidence estimates produced by sequential Monte Carlo sweeps.
Source code for reproduction of figures and results in this paper is available at \url{https://github.com/plai-group/stdr}.
